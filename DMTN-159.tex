\documentclass[DM,lsstdraft,authoryear,toc]{lsstdoc}
% lsstdoc documentation: https://lsst-texmf.lsst.io/lsstdoc.html
\input{meta}

% Package imports go here.

% Local commands go here.

%If you want glossaries
%\input{aglossary.tex}
%\makeglossaries

\title{Report on Operations Rehearsl \#2}

% Optional subtitle
% \setDocSubtitle{A subtitle}

\author{%
Robert Gruendl (et al.?)
}

\setDocRef{DMTN-159}
\setDocUpstreamLocation{\url{https://github.com/lsst-dm/dmtn-159}}

\date{\vcsDate}

% Optional: name of the document's curator
% \setDocCurator{The Curator of this Document}

\setDocAbstract{%
This report describes the activities and lessons learned in the execution of Operations Rehearsal \# 2.
}

% Change history defined here.
% Order: oldest first.
% Fields: VERSION, DATE, DESCRIPTION, OWNER NAME.
% See LPM-51 for version number policy.
\setDocChangeRecord{%
  \addtohist{1}{2020-08-10}{Unreleased.}{Robert Gruendl}
}


\begin{document}

% Create the title page.
\maketitle
% Frequently for a technote we do not want a title page  uncomment this to remove the title page and changelog.
% use \mkshorttitle to remove the extra pages

% ADD CONTENT HERE
% You can also use the \input command to include several content files.

\input body.tex

\appendix
% Include all the relevant bib files.
% https://lsst-texmf.lsst.io/lsstdoc.html#bibliographies
\section{References} \label{sec:bib}
\renewcommand{\refname}{} % Suppress default Bibliography section
\bibliography{local,lsst,lsst-dm,refs_ads,refs,books}

% Make sure lsst-texmf/bin/generateAcronyms.py is in your path
\section{Acronyms} \label{sec:acronyms}
\addtocounter{table}{-1}
\begin{longtable}{p{0.145\textwidth}p{0.8\textwidth}}\hline
\textbf{Acronym} & \textbf{Description}  \\\hline

 &  \\\hline
CCD & Charge-Coupled Device \\\hline
COVID & COrona VIrus Disease \\\hline
ComCam & The commissioning camera is a single-raft, 9-CCD camera that will be installed in LSST during commissioning, before the final camera is ready. \\\hline
DAQ & Data Acquisition System \\\hline
DB & DataBase \\\hline
DM & Data Management \\\hline
DMTN & DM Technical Note \\\hline
GB & Gigabyte \\\hline
IT & Information Technology \\\hline
L3 & Lens 3 \\\hline
LDM & LSST Data Management (Document Handle) \\\hline
LHN & long haul network \\\hline
LSST & Legacy Survey of Space and Time (formerly Large Synoptic Survey Telescope) \\\hline
MB & MegaByte \\\hline
N2 & Nitrogen \\\hline
NCSA & National Center for Supercomputing Applications \\\hline
OCS & Observatory Control System \\\hline
OPS & Operations \\\hline
PST & Project Science Team \\\hline
QA & Quality Assurance \\\hline
RSP & Rubin Science Platform \\\hline
USDF & United States Data Facility \\\hline
\end{longtable}

% If you want glossary uncomment below -- comment out the two lines above
%\printglossaries





\end{document}
